\documentclass[a4paper,10pt]{jarticle}


\pagestyle{plain}

%%%%%%    TEXT START    %%%%%%
\begin{document}
\section{AzLib 概要}

\subsection{ライセンス}

\subsection{各ディレクトリの概要}

\subsection{AzLib の利用法}
\subsubsection{ヘッダファイルのインクルード}
\subsubsection{make によるコンパイル方法}

\subsection{AzLib の開発方法}
\subsubsection{CVS によるソース管理}
\subsubsection{ML, Web による情報交換}
\subsubsection{各ファイル内ヘッダの記述規則}

%%%%%%%%%%%%%%%%%%%%%%%%%%%%%%%%%%%%%%%%%%%%%%%%%%

\section{リターンコード RC と関連マクロ (rc.h)}
対象ソースコード : \verb|rc.c|

\subsection{リターンコード RC}
\subsubsection{RC の設計思想}

\subsection{RC\_TRY(), RC\_TRY\_MAIN() マクロ}

\subsection{RC\_NEG\_CHK(), RC\_NEG\_ZERO\_CHK(), RC\_NULL\_CHK() マクロ}
\subsubsection{RC\_*() 系マクロの実装}   <--- 関数型マクロ等

\subsection{rc\_* 系関数}

%%%%%%%%%%%%%%%%%%%%%%%%%%%%%%%%%%%%%%%%%%%%%%%%%%

\section{数学関連ユーティリティー(math\_utl.h)}
対象ソースコード : \verb|math_utl.c| , \verb|integration.c|

\subsection{行列および多次元配列}
\subsubsection{配列の動的確保と解放(allocate*(), free*() 系関数)}
\subsubsection{行列計算用関数}
\paragraph{配列処理の実装}   <-- ポインタ関連
\paragraph{行列計算プログラムの実装と高速化技法}   <---ループアンローリング等

\subsection{二次元,三次元ベクトル}
\subsubsection{TRANSLATION3D 系データ構造}
\subsubsection{二次元,三次元ベクトル処理関数}

\subsection{有限要素内の数値積分(Gauss\_const\_*() 系関数)}

\subsection{ABS\_TOL, REL\_TOL マクロ}

\subsection{その他の数学関連関数,マクロ}

%%%%%%%%%%%%%%%%%%%%%%%%%%%%%%%%%%%%%%%%%%%%%%%%%%

\section{バイナリーファイル入出力(bin\_io.h)}
対象ソースコード : \verb|bin_io.c|

\subsection{ビッグエンディアン,リトルエンディアン(ENDIAN)}

\subsection{バイナリーファイル入出力関数}

%%%%%%%%%%%%%%%%%%%%%%%%%%%%%%%%%%%%%%%%%%%%%%%%%%

\section{メール送信ユーティリティー(mail\_utl.h)}
対象ソースコード : \verb|mail_utl.c|

\subsection{メール送信関数(send\_mail() 関数)}
\subsubsection{メール送信関数の実装}   <--- \verb|system()| 関数の使い方

%%%%%%%%%%%%%%%%%%%%%%%%%%%%%%%%%%%%%%%%%%%%%%%%%%

\section{文字列処理ユーティリティー(string\_utl.h)}
対象ソースコード : \verb|string_utl.c| , \verb|idc.c|

\subsection{文字列の管理}
\subsubsection{文字列管理構造体(STRING\_ARRAY)}
\subsubsection{文字列の管理関数(*\_string\_array() 系関数)}

\subsection{任意データ型の処理}
\subsubsection{任意のデータ型を保持する共用体(IDC)}
\subsubsection{任意データ型の読み込み(fgetidc(), sgetidc() 関数)}

\subsection{ログファイル出力(logprintf() 関数)}
\subsubsection{ログファイル出力の実装}  <--- 可変引数の処理

%%%%%%%%%%%%%%%%%%%%%%%%%%%%%%%%%%%%%%%%%%%%%%%%%%

\section{ポケコン風数式処理(pokecom.h)}
対象ソースコード : \verb|pokecom.c|

\subsection{変数管理構造体(POKE\_VARIABLE)}

\subsection{変数管理用関数}

\subsection{数式処理(eval\_expr() 関数)}
\subsubsection{数式処理の実装}   <-- 関数の再帰呼び出し等

%%%%%%%%%%%%%%%%%%%%%%%%%%%%%%%%%%%%%%%%%%%%%%%%%%

\section{スカイライン型行列処理(sky\_cholesky.h, sky\_crout.h)}
対象ソースコード : \verb|sky_cholesky.c| , \verb|sky_crout.c|

\subsection{LU 分解とスカイライン法}
  <-- 改訂コレスキー分解,クラウト分解とスカイライン型行列の必要性

\subsection{スカイライン型行列のデータ構造(n, index1, index2, array)}
\subsubsection{対称行列の場合}
\subsubsection{非対称行列の場合}

\subsection{スカイライン型行列の確保と解放(*\_alloc(), *\_free() 系関数)}

\subsection{スカイライン型行列への代入,修正}

\subsection{スカイライン型行列の LU 分解と前進・後退代入}
\subsubsection{LU 分解の実装と高速化技法}  <-- 三段ループアンローリング等

%%%%%%%%%%%%%%%%%%%%%%%%%%%%%%%%%%%%%%%%%%%%%%%%%%

\section{非ゼロ型行列処理(nonzero\_cg.h)}
対象ソースコード : \verb|nonzero_cg.c|

\subsection{CG 法と非ゼロ型行列}
   <-- CG 法による連立一次方程式の解法と非ゼロ型行列の必要性

\subsection{非ゼロ型行列用データ構造}
\subsubsection{行圧縮方式(NONZERO\_MATRIX)}
\subsubsection{3x3 ブロック行圧縮方式(NONZERO\_MATRIX3)}

\subsection{非ゼロ型行列の確保と解法}

\subsection{非ゼロ型行列に関連する各種演算処理}

\subsection{非ゼロ型行列に対する各種 CG 法}
\subsubsection{対称行列用 ICCG 法(iccg\_nonzero3s() 関数)}
\subsubsection{非対称用 CG 法(cg\_nonzero1a\_*() 系関数)}

%%%%%%%%%%%%%%%%%%%%%%%%%%%%%%%%%%%%%%%%%%%%%%%%%%

\section{有限要素法用データ構造(fem\_struct.h)}
対象ソースコード : \verb|fem_struct.c| , \verb|fem_struct_print.c| ,
                   \verb|fem_struct_binio.c| , \verb|fem_utl.|c ,
                   \verb|fem_renumber.c| , \verb|element_volume.c| , 
                   \verb|extract_surface.c| ,
                   \verb|interpolation.c| , \verb|pressure_force.c|

\subsection{各種データ構造(FEM\_NODE, FEM\_ELEMENT 等)}
\subsubsection{付加情報(i\_info, d\_info) の利用法}
\subsubsection{データの初期化と解放(init\_fem\_*(), free\_fem\_*() 系関数)}
\subsubsection{ファイル出力用関数(print\_fem\_*() 系関数)}

\subsection{各種データの配列管理用構造体(FEM\_NODE\_ARRAY, FEM\_ELEMENT\_ARRAY 等)}
\subsubsection{配列の確保と解放(allocate\_fem\_*\_array(), realloc\_fem\_*\_array(), clean\_fem\_*\_array(), free\_fem\_*\_array 系関数)}
\subsubsection{配列内の特定要素の検索(sort\_fem\_*\_array(), search\_fem\_*\_*() 系関数)}
\subsubsection{配列のファイル出力用関数(print\_fem\_*\_array() 系関数)}
\subsubsection{配列に対するその他の操作}

\subsection{節点のリナンバリング}
\subsubsection{AzLib における節点自由度と連立一次方程式の変数との対応}
  <-- \verb|FEM_NODE| の \verb|renum| と \verb|node_renum_index()|,
  \verb|node_renum_index_cache()| の解説
\paragraph{node\_renum\_index\_cache() の実装}  <-- キャッシュアルゴリズム
\subsubsection{リナンバリング処理(dummy\_renumber(), fem\_renumber0() 関数)}
\paragraph{リナンバリングアルゴリズムの実装}  <-- グラフ処理等

\subsection{体積計算(element\_volume(), total\_element\_volume() 関数}
\subsubsection{体積計算:ヤコビアンの積分計算の実装}
    <-- ヤコビアン計算,ガウスポイント等

\subsection{要素外表面抽出,要素境界面抽出(extract\_surface(), extract\_divider() 関数)}
\subsubsection{要素外表面抽出,要素境界面抽出の実装}  <-- ハッシュテーブル等

\subsection{有限要素の再帰的細分割(cut\_element() 関数)}
\subsubsection{有限要素の再帰的細分割の実装}
   <-- 独自のデータ構造,ハッシュテーブルの利用等

\subsection{要素内での内挿関数とその勾配(set\_N\_vector(), set\_dN\_matrix() 関数)}
\subsubsection{要素の中心,各節点のアイソパラメトリック座標(set\_local\_node\_points(),
set\_center\_point() 関数)}

\subsection{表面分布力の節点荷重への振り分け(pressure\_force(), traction\_force() 関数)}

\subsection{その他の有限要素法関連関数}

%%%%%%%%%%%%%%%%%%%%%%%%%%%%%%%%%%%%%%%%%%%%%%%%%%

\section{ANSYS 用データ入出力(ansys\_component.h)}
対象ソースコード : \verb|ansys_component.c|

\subsection{ANSYS データファイルからの入力(ansys\_input\_*() 系関数}

\subsection{ANSYS データファイルへの出力(ansys\_output\_*() 系関数}

%%%%%%%%%%%%%%%%%%%%%%%%%%%%%%%%%%%%%%%%%%%%%%%%%%

\section{SYSNOISE 用データ入出力(sysnoise\_component.h)}
対象ソースコード : \verb|sysnoise_component.c|

\subsection{SYSNOISE データファイルからの入力(sysnoise\_input\_*() 系関数}

\subsection{SYSNOISE データファイルへの出力(sysnoise\_output\_*() 系関数}

%%%%%%%%%%%%%%%%%%%%%%%%%%%%%%%%%%%%%%%%%%%%%%%%%%

\section{STL 用データ入出力(stl\_utl.h)}
対象ソースコード : \verb|stl_utl.c|

\subsection{STL ファイルからの入力(input\_stl\_3dface*() 系関数}

\subsection{STL ファイルへの出力(output\_stl\_3dface*() 系関数}

%%%%%%%%%%%%%%%%%%%%%%%%%%%%%%%%%%%%%%%%%%%%%%%%%%

\section{STL 用データ入出力(dxf\_utl.h)}
対象ソースコード : \verb|dxf_utl.c|

\subsection{DXF ファイルからの入力(input\_dxf*() 系関数}

\subsection{DXF ファイルへの出力(output\_dxf*() 系関数}

%%%%%%%%%%%%%%%%%%%%%%%%%%%%%%%%%%%%%%%%%%%%%%%%%%

\section{EPS 用データ出力(eps\_utl.h)}
対象ソースコード : \verb|eps_utl.c|

\subsection{EPS ファイルへの出力(output\_eps() 関数)}

%%%%%%%%%%%%%%%%%%%%%%%%%%%%%%%%%%%%%%%%%%%%%%%%%%

\section{Nastran 用データ入出力(nst\_component.h)}
対象ソースコード : \verb|nst_blk_llio.c| , \verb|nst_blk_component.c| ,
  \verb|nst_f06_component.c| , \verb|nst_pch_component.c|

\subsection{Nastran BULK データからの入力(nst\_input\_*() 系関数)}
\subsubsection{Nastran BULK データ用低レベル入力関数(nst\_blk\_get\_tok() 関数等)}

\subsection{Nastran BULK データへの出力(nst\_output\_*() 系関数)}
\subsubsection{Nastran BULK データ用低レベル出力関数(nst\_print() 関数等)}

\subsection{Nastran f06 ファイルからの入力(nst\_input\_*() 系関数)}
\subsubsection{Nastran f06 用低レベル入力関数(read\_f06*() 系関数)}

\subsection{Nastran Panch ファイルからの入力(read\_pch\_* 系関数)}

%%%%%%%%%%%%%%%%%%%%%%%%%%%%%%%%%%%%%%%%%%%%%%%%%%

\section{有限要素弾性解析(fem\_solver.h)}
対象ソースコード : \verb|fem_solver.c| , \verb|fem_solver_6dof.c| , \verb|nl_solver.c| , \verb|eigen_solver.c|

\subsection{線形弾性解析(fem\_solver.c)}

\subsection{線形弾性解析の1節点6自由度版(fem\_solver\_6dof.c)}

\subsection{大変形解析(nl\_solver.c)}

\subsection{固有振動解析(eigen\_solver.c)}

%%%%%%%%%%%%%%%%%%%%%%%%%%%%%%%%%%%%%%%%%%%%%%%%%%

\section{流れ場解析(flow\_solver.h)}
対象ソースコード : \verb|flow_solver.c|

\subsection{流れ場解析}

\subsection{流れ場の最適化関連関数}

%%%%%%%%%%%%%%%%%%%%%%%%%%%%%%%%%%%%%%%%%%%%%%%%%%

\section{力法による最適化解析(optimization\_utl.h)}
対象ソースコード : \verb|optimization_utl.c|

\subsection{速度場解析(speed\_analysis() 関数}

\subsection{各種評価関数と評価関数の変動予測(*\_functional() 系関数, functional\_variation())}

\subsection{各種評価関数の感度を節点荷重に変換(shape\_trac\_*() 系関数)}

\subsection{その他の最適化解析関連関数}


\end{document}


%%%%%%%%%%%%%% .latex2html の内容%%%%%%%%%%%%%%%%%%%%%%%

$default_language = 'japanese';
$TITLES_LANGUAGE = 'japanese';
$LOCAL_ICONS = 1;
$ADDRESS = '
<p> Copyright (C) 2001 AzLib Developers Group </p>
<p> Written by &lt;Kenzen TAKEUCHI&gt; </p>
<p> Thanks to following contributers. &lt;  &gt;</p>
'; 

